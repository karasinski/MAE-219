%%%%%%%%%%%%%%%%%%%%%%%%%%% asme2ej.tex %%%%%%%%%%%%%%%%%%%%%%%%%%%%%%%
% Template for producing ASME-format journal articles using LaTeX    %
% Written by   Harry H. Cheng, Professor and Director                %
%              Integration Engineering Laboratory                    %
%              Department of Mechanical and Aeronautical Engineering %
%              University of California                              %
%              Davis, CA 95616                                       %
%              Tel: (530) 752-5020 (office)                          %
%                   (530) 752-1028 (lab)                             %
%              Fax: (530) 752-4158                                   %
%              Email: hhcheng@ucdavis.edu                            %
%              WWW:   http://iel.ucdavis.edu/people/cheng.html       %
%              May 7, 1994                                           %
% Modified: February 16, 2001 by Harry H. Cheng                      %
% Modified: January  01, 2003 by Geoffrey R. Shiflett                %
% Butchered: October 15, 2014 by John Karasinski                     %
% Use at your own risk, send complaints to /dev/null                 %
%%%%%%%%%%%%%%%%%%%%%%%%%%%%%%%%%%%%%%%%%%%%%%%%%%%%%%%%%%%%%%%%%%%%%%

%%% use twocolumn and 10pt options with the asme2ej format
\documentclass[twocolumn,10pt]{asme2ej}

\usepackage{epsfig} %% for loading postscript figures
\usepackage{amsmath}
\usepackage{graphicx}
\usepackage{grffile}
\usepackage{pdfpages}
\usepackage{algpseudocode}

% Default fixed font does not support bold face
\DeclareFixedFont{\ttb}{T1}{txtt}{bx}{n}{12} % for bold
\DeclareFixedFont{\ttm}{T1}{txtt}{m}{n}{12}  % for normal

% Custom colors
\usepackage{color}
\usepackage{listings}
\usepackage{framed}
\usepackage{caption}
\captionsetup[lstlisting]{font={small,tt}}

\definecolor{mygreen}{rgb}{0,0.6,0}
\definecolor{mygray}{rgb}{0.5,0.5,0.5}
\definecolor{mymauve}{rgb}{0.58,0,0.82}

\lstset{ %
  backgroundcolor=\color{white},   % choose the background color; you must add \usepackage{color} or \usepackage{xcolor}
  basicstyle=\ttfamily\footnotesize, % the size of the fonts that are used for the code
  breakatwhitespace=false,         % sets if automatic breaks should only happen at whitespace
  % breaklines=true,                 % sets automatic line breaking
  captionpos=b,                    % sets the caption-position to bottom
  commentstyle=\color{mygreen},    % comment style
  deletekeywords={...},            % if you want to delete keywords from the given language
  escapeinside={\%*}{*)},          % if you want to add LaTeX within your code
  extendedchars=true,              % lets you use non-ASCII characters; for 8-bits encodings only, does not work with UTF-8
  frame=single,                    % adds a frame around the code
  keepspaces=true,                 % keeps spaces in text, useful for keeping indentation of code (possibly needs columns=flexible)
  columns=flexible,
  keywordstyle=\color{blue},       % keyword style
  language=Python,                 % the language of the code
  morekeywords={*,...},            % if you want to add more keywords to the set
  numbers=left,                    % where to put the line-numbers; possible values are (none, left, right)
  numbersep=5pt,                   % how far the line-numbers are from the code
  numberstyle=\tiny\color{mygray}, % the style that is used for the line-numbers
  rulecolor=\color{black},         % if not set, the frame-color may be changed on line-breaks within not-black text (e.g. comments (green here))
  showspaces=false,                % show spaces everywhere adding particular underscores; it overrides 'showstringspaces'
  showstringspaces=false,          % underline spaces within strings only
  showtabs=false,                  % show tabs within strings adding particular underscores
  stepnumber=1,                    % the step between two line-numbers. If it's 1, each line will be numbered
  stringstyle=\color{mymauve},     % string literal style
  tabsize=4,                       % sets default tabsize to 2 spaces
}

\title{Case Study \# 4: Linear 1D Transport Equation}

\author{John Karasinski
    \affiliation{
  Graduate Student Researcher\\
  Center for Human/Robotics/Vehicle Integration and Performance\\
  Department of Mechanical and Aerospace Engineering\\
  University of California\\
  Davis, California 95616\\
    Email: karasinski@ucdavis.edu
    }
}

\begin{document}
\maketitle

%%%%%%%%%%%%%%%%%%%%%%%%%%%%%%%%%%%%%%%%%%%%%%%%%%%%%%%%%%%%%%%%%%%%%%
\section{Problem Description}
The transport of various scalar quantities in flows (e.g. species mass fraction, temperature) can be modeled using a linear convection-diffusion equation (presented here in a 1-D form),
\begin{equation}
\frac{\partial \phi}{\partial t} + u \frac{\partial \phi}{\partial x} = D \frac{\partial \phi^2}{\partial x^2},
\end{equation}
where $\phi$ is the transported scalar, $u$ and $D$ are known parameters (flow velocity and diffusion coefficient respectively). The purpose of this case study is to investigate the behavior of various numerical solutions of this equation.

This case study focuses on the solution of the 1-D linear transport equation (above)
for $x$ $\in$ $[0, L]$ and $t$ $\in$ $[0, \tau ]$ (where $\tau = 1/k^2D$) subject to periodic boundary conditions and the following initial condition
\begin{equation}
\phi(x, 0) = sin(kx),
\end{equation}
\noindent with $k = 2\pi/L$ and $L=1$ m. The convection velocity is $u = 0.2$ m/s, and the diffusion coefficient is $D = 0.005$ m$^2$/s.

This problem has an analytical solution \cite{analytic_citation},
\begin{equation}
\label{analytic_solution}
\Phi(x, t) = exp(-k^2Dt) sin[k(x-ut)].
\end{equation}

Numerical solutions of this problem were created using the following schemes:
\begin{enumerate}
\item FTCS (Explicit) - Forward-Time and central differencing for both the convective flux and the diffusive flux.
\item Upwind - Finite Volume method: Explicit (forward Euler), with the convective flux treated using
the basic upwind method and the diffusive flux treated using central differencing.
\item Trapezoidal - (AKA Crank-Nicholson).
\item QUICK - Finite Volume method: Explicit, with the convective flux treated using the QUICK method and the diffusive flux treated using central differencing.
\end{enumerate}

The following cases are considered:
\begin{equation*}
(C, s) \in \{(0.1, 0.25), (0.5, 0.25), (2, 0.25), (0.5, 0.5), (0.5, 1)\}
\end{equation*}
\noindent where $C = u\Delta t/\Delta x$ and $s = D\Delta t/\Delta x^2$. A uniform mesh was generated with spacing $\Delta x$ for all solvers and cases. The stability and accuracy of these schemes was investigated.

%%%%%%%%%%%%%%%%%%%%%%%%%%%%%%%%%%%%%%%%%%%%%%%%%%%%%%%%%%%%%%%%%%%%%%
\section{Numerical Solution Approach}

Three explicit schemes and one implicit scheme were developed to investigate the five considered cases. These are an explicit FTCS scheme, an upwind finite volume scheme, an implicit trapezoidal scheme, and a QUICK finite volume scheme. Each case makes use of $C$ and $s$ to to compute the spatial ($\Delta x = C D / u s$) and temporal ($\Delta t = C \Delta x / u$) discretizations.

\subsection{FTCS Scheme}
The first scheme involves using forward-time and central differencing (FTCS) for both the convective flux and the diffusive flux and yields second-order convergence in space and first-order convergence in time. In order to implement this method, the domain of the problem must be discretized. This method calculates the state of the system at a later time from the state of the system at the current time, and is thus an explicit method. For the 1-D transport equation on a uniform grid, the state $\phi$ at grid point $i$ and time step $f$ can be calculated by the following equation,

\begin{equation}
\label{ftcs_eqn}
\begin{split}
\phi_i ^f = & \left(\frac{D \Delta t}{\Delta x^2} + \frac{u \Delta t}{2 \Delta x}\right) \phi_{i-1} ^{f-1} + \\
            & \left(1-\frac{2 D\Delta t}{\Delta x^2}\right) \phi_i ^{f-1} + \\
            & \left(\frac{D \Delta t}{\Delta x^2} - \frac{u \Delta t}{2 \Delta x}\right) \phi_{i+1} ^{f-1}.
\end{split}
\end{equation}

\noindent To impose the periodic boundary condition, the last node in the domain reaches around to the second node, while the first node is set equivalent to the last node. This scheme is numerically stable as long as the following conditions are satisfied:
\begin{equation}
\label{FTCS_stability}
C \leq \sqrt{2su} \mbox{ and } s \leq \frac{1}{2}.
\end{equation}

\begin{figure}[t]
\begin{center}
\includegraphics[width=0.4\textwidth]{figure/FTCS.png}
\caption{The 1-D FTCS scheme interpolates between the two nearby grid points~\cite{analytic_citation}}
\label{FTCS_scheme}
\end{center}
\end{figure}

\begin{figure}[t]
\begin{center}
\includegraphics[width=0.4\textwidth]{figure/Upwind.png}
\caption{Upwind scheme's interpolation for the diffusive flux~\cite{analytic_citation}}
\label{Upwind_scheme}
\end{center}
\end{figure}

\begin{figure}[t]
\begin{center}
\includegraphics[width=0.4\textwidth]{figure/Trapezoidal (time derivative).png}
\caption{Trapezoidal scheme's interpolation for the time derivative~\cite{analytic_citation}}
\label{Trapezoidal_scheme}
\end{center}
\end{figure}

\subsection{Upwind Scheme}
The second scheme is an explicit upwind finite volume method. For this method the convective flux is treated using the basic upwind method and the diffusive flux treated using central differencing. This is a second-order scheme which uses a three point backward difference, as described below

\begin{equation}
\label{upwind_eqn}
\begin{split}
\phi_i ^f = & \left( \frac{D \Delta t}{\Delta x^2} \right) \left[ \phi_{i+1} ^{f-1} - 2 \phi_{i} ^{f-1} + \phi_{i-1} ^{f-1} \right] - \\
            & \left( \frac{u \Delta t}{2 \Delta x} \right) \left[ 3 \phi_{i} ^{f-1} - 4 \phi_{i-1} ^{f-1} + \phi_{i-2} ^{f-1} \right] + \\
            & \phi_i ^{f-1}.
\end{split}
\end{equation}

\noindent The upwind method is more stable than the FTCS scheme, and unlike the FTCS scheme, the stability of the Upwind scheme does not depend on $u$. To impose the periodic boundary condition, the last node and the second node in the domain reach across the edge of the domain, while the first node is set equivalent to the last node. This scheme is numerically stable as long as the following condition is satisfied:
\begin{equation}
\label{Upwind_stability}
C + 2s \leq 1.
\end{equation}

\subsection{Trapezoidal (Crank-Nicholson) Scheme}
The Trapezoidal scheme is a finite difference method which is implicit and unconditionally stable. This method is an equally weighted average of the explicit and implicit central difference solutions. This is accomplished by setting $\theta = \frac{1}{2}$ in the following equation:
\begin{equation*}
\begin{split}
\theta & \left[\left(C-s\right)\phi^{f+1}_{i+1} + \left(\frac{1}{\theta}+2s\right)\phi^{f+1}_{i} - \left(s+C\right)\phi^{f+1}_{i-1}\right] = \\
(1-\theta) & \left[\left(-C+s\right)\phi^{f}_{i+1} + \left(\frac{1}{1-\theta}-2s\right)\phi^{f}_{i} + \left(s+C\right)\phi^{f}_{i-1}\right].
\end{split}
\end{equation*}

\noindent This leads to an implicit method, a system of algebraic equations must be solved to find values of the transported scalar for the next time step. This problem requires the solution of a nearly tridiagonal matrix, with the exception of the top right and bottom left corners, which are set to impose the periodic boundary condition~\cite{hogarth1990comparative}.

The following set of equations must be solved to advance the solution to the next time step:
\begin{equation}
\begin{split}
\begin{bmatrix}
   { b } & { c } & {   }  & {   }  & { a } \\
   { a } & { b } & { c }  & {   }  & {   } \\
   {   } & { a } & { b }  & { c }  & {   } \\
   {   } & \ddots & \ddots & \ddots & {  } \\
   {   } & {   } & { a }  & { b } & { c } \\
   { c } & {   } & {   }  & { a }  & { b } \\
\end{bmatrix}
\begin{bmatrix}
   {\phi_1^{f} }   \\
   {\phi_2^{f} }   \\
   {\phi_3^{f} }   \\
   \vdots   \\
   {\phi_{i-1}^{f} } \\
   {\phi_{i}^{f} } \\
\end{bmatrix}
& =
\begin{bmatrix}
   {RHS_1^{f} }   \\
   {RHS_2^{f} }   \\
   {RHS_3^{f} }   \\
   \vdots   \\
   {RHS_{i-1}^{f} } \\
   {RHS_{i}^{f} } \\
\end{bmatrix},
\end{split}
\end{equation}
where $a = -A - B$, $b = 1 + 2 A $, and $c = -A + B$, and where
\begin{equation}
A = \frac{D \Delta t }{2 \Delta x^2} \mbox{ and } B = \frac{u \Delta t }{4 \Delta x},
\end{equation}
\noindent The right hand side of the equation is a linear combination of the solutions from the previous time step,
\begin{equation}
\begin{split}
RHS_i^f = & A (\phi_{i+1}^{f-1} - 2 \phi_i^{f-1} + \phi_{i-1}^{f-1}) - \\
        & B (\phi_{i+1}^{f-1} - \phi_{i-1}^{f-1}) + \\
        & \phi_i^{f-1}.
\end{split}
\end{equation}

\subsection{QUICK Scheme}

\begin{figure}[tbh]
\begin{center}
\includegraphics[width=0.4\textwidth]{figure/QUICK.png}
\caption{The QUICK scheme interpolates between two quadratic equations \cite{analytic_citation}}
\label{QUICK_scheme}
\end{center}
\end{figure}

The Quadratic Upstream Interpolation for Convective Kinematics (QUICK) method is an explicit method which uses three point upstream weighted quadratic interpolation for cell phase values (see Figure~\ref{QUICK_scheme}). Here the convective flux is treated using the QUICK method, while the diffusive flux treated using central differencing. This scheme is second-order accurate for the finite difference model \cite{chen1992advection}. This can be implemented with the following equation:
\begin{equation}
\label{quick_eqn}
\begin{split}
\phi_i ^f = & \left( \frac{D \Delta t}{\Delta x^2} \right) \left[ \phi_{i+1} ^{f-1} - 2 \phi_{i} ^{f-1} + \phi_{i-1} ^{f-1} \right] - \\
            & \left( \frac{u \Delta t}{8 \Delta x} \right) \left[ 3 \phi_{i+1} ^{f-1}  + 3 \phi_{i} ^{f-1}  + \phi_{i-2} ^{f-1} - 7 \phi_{i-1} ^{f-1} \right] + \\
            & \phi_i ^{f-1}
\end{split}
\end{equation}

To impose the periodic boundary condition, the last node and the second node in the domain reach across the edge of the domain, while the first node is set equivalent to the last node. This scheme is numerically stable under the following condition:
\begin{equation}
\label{QUICK_stability}
C \leq min(2 - 4 s, \sqrt{2s}).
\end{equation}

%%%%%%%%%%%%%%%%%%%%%%%%%%%%%%%%%%%%%%%%%%%%%%%%%%%%%%%%%%%%%%%%%%%%%%
\section{Results Discussion}
\subsection{Stability}
For the results below, cases 1, 2, 3, 4, 5 refer to $(C, s) = (0.1, 0.25), (0.5, 0.25), (2, 0.25), (0.5, 0.5), (0.5, 1)$, respectively. The stability for each scheme was investigated for each case. The stability criteria for the FTCS, Upwind, and QUICK schemes can be found as Equations~\ref{FTCS_stability},~\ref{Upwind_stability}, and~\ref{QUICK_stability}~\cite{LectureA}. For the cases considered, FTCS was least stable, the Upwind and QUICK schemes were effectively equally stable, while the Trapezoidal scheme is inherently stable. For the full results, see Table~\ref{stability_table}.

\begin{table}[tb]
\begin{center}
\begin{tabular}{|r | l l l l|}
\hline
Case & FTCS & Upwind & Trap & QUICK  \\
\hline
1 & True  & True  & True & True  \\
2 & False & True  & True & True  \\
3 & False & False & True & False \\
4 & False & False & True & False \\
5 & False & False & True & False \\
\hline
\end{tabular}
\caption{Stability results for each case and method}
\label{stability_table}
\end{center}
\end{table}

\subsection{NRMS}
The computational result for the 1-D linear convection-diffusion equation can be compared to the analytical result above, Equation~\ref{analytic_solution}. The Root Mean Square error,
\begin{equation}
RMSE = \sqrt{\frac{1}{N}\sum\limits_{i=1}^N[\phi_i - \phi^*_i]^2},
\end{equation}
and the Normalized Root Mean Square error,
\begin{equation}
NRMS = \dfrac{RMSE}{max(\phi^*)-min(\phi^*)},
\end{equation}
can be calculated. Here $\phi_i$ is the computational result for the the transported scalar for each point on the 1-D domain, $\phi^*_i$ is the analytical solution, and $N$ is the number of points on the 1-D domain. The $NRMS$ for each case is expressed as a percentage, where lower values indicate a result closer to the analytic solution. For the complete NRMS results for each case and scheme, see Table~\ref{rms_table}.

\begin{table}[tb]
\begin{center}
\begin{tabular}{|r | l l l l|}
\hline
Case & FTCS & Upwind & Trap & QUICK  \\
\hline
1 & 7.23E-03 & 9.68E-03 & 2.42E-02 & 7.67E-03 \\
2 & 2.23E-01 & 2.89E-01 & 1.30E-01 & 2.32E-01 \\
3 & 8.26E+00 & 3.64E+01 & 7.72E-01 & 1.24E+01 \\
4 & 1.06E-01 & 6.99E+22 & 4.56E-02 & 1.10E-01 \\
5 & 1.10E+61 & 1.28E+97 & 2.14E-02 & 7.37E+71 \\
\hline
\end{tabular}
\caption{NRMS results for each case and method (each value is expressed as a percentage)}
\label{rms_table}
\end{center}
\end{table}

The lowest NRMS error is found by using the FTCS scheme under Case 1. Despite this, the FTCS case quickly blows up for Cases 3 and 5 due to numerical instability. The QUICK method performs similarly, with approximately the same error for all cases. The Trapezoidal method's unconditional stability leads to it performing best for all cases aside from Case 1.

The cases of large NRMS arise from the loss of stability in the scheme. The effect of instability can be seen quite clearly in Figure~\ref{QUICK_transition}. Making use of Equation~\ref{QUICK_stability} with values $C=0.5$ and $s=1$, for instance, one can see that
\begin{equation}
\begin{split}
C   & \leq min(2 - 4 s, \sqrt{2s}) \\
0.5 & \leq min(-2, \sqrt 2) \\
0.5 & \leq -2
\end{split}
\end{equation}
is false. This leads to the instability and resultant large NRMS of 4.70E+85.

\subsection{Effective Order}
\begin{figure}[tb]
\begin{center}
\includegraphics[width=0.5\textwidth]{../code/figures/Order QUICK.pdf}
\caption{Effective order of the QUICK method}
\label{QUICK_order}
\end{center}
\end{figure}
The effective order of each method was calculated by fitting the NRMS for cases within the stability region of each method. The effective order was found by fitting with a linear function against a log log plot of the NRMS versus the $\Delta x$ and $\Delta t$, see Figure~\ref{QUICK_order} for an example. The slope of the fit estimates the order of accuracy of the method. The effective orders of accuracy for the FTCS, Trapezoidal, and QUICK methods are approximately 2 for the spatial dimension and approximately 1 for the temporal dimension. The Upwind method is approximately first order accurate in the spatial dimension. For full results, see Table~\ref{effective_order_table}.

\begin{table}[tb]
\begin{center}
\begin{tabular}{|r | r | r |}
\hline
Method & $\Delta x$ & $\Delta t$ \\
\hline
FTCS    & 2.06 & 1.06 \\
Upwind  & 1.08 & 0.55 \\
Trap    & 1.97 & 0.99 \\
QUICK   & 2.32 & 1.16 \\
\hline
\end{tabular}
\caption{Effective order of each method for $\Delta x$ and $\Delta t$}
\label{effective_order_table}
\end{center}
\end{table}

%%%%%%%%%%%%%%%%%%%%%%%%%%%%%%%%%%%%%%%%%%%%%%%%%%%%%%%%%%%%%%%%%%%%%%
\section{Conclusion}
Only the first case led to stable solutions for each scheme. The results from this case can be seen in Figure~\ref{case_1_results}. The error between each scheme's result and the analytic solution can be seen in Figure~\ref{case_1_error}. This error shows that the three explicit methods can perform better than the implicit method for low CFL numbers, though for larger CFL numbers the opposite is also true.

While the FTCS and QUICK schemes produced the lowest error in the majority of the considered cases, the unconditional stability of the Trapezoidal method makes it the more reliable method if the $C$ and $s$ values cannot be chosen freely. The spatial accuracy of the schemes implemented here can be improved by including more data points in their derivation, which would offer a more accurate finite difference stencil for the approximation of spatial derivative.

\begin{figure}[thb]
\begin{center}
\includegraphics[width=0.5\textwidth]{../code/figures/C=0.1 s=0.25.pdf}
\caption{Results of each scheme for Case 1}
\label{case_1_results}
\end{center}
\end{figure}

\begin{figure}[thb]
\begin{center}
\includegraphics[width=0.5\textwidth]{../code/figures/Error C=0.1 s=0.25.pdf}
\caption{Error for each scheme for Case 1}
\label{case_1_error}
\end{center}
\end{figure}

%%%%%%%%%%%%%%%%%%%%%%%%%%%%%%%%%%%%%%%%%%%%%%%%%%%%%%%%%%%%%%%%%%%%%%
\nocite{*}
\bibliographystyle{asmems4}
\bibliography{asme2e}

\begin{figure}[thb]
\begin{center}
\includegraphics[width=0.5\textwidth]{../code/figures/QUICK 0.1 0.25.pdf}
\includegraphics[width=0.5\textwidth]{../code/figures/QUICK 0.5 0.25.pdf}
\includegraphics[width=0.5\textwidth]{../code/figures/QUICK 0.5 1.pdf}
\caption{QUICK method's transition into instability}
\label{QUICK_transition}
\end{center}
\end{figure}

%%%%%%%%%%%%%%%%%%%%%%%%%%%%%%%%%%%%%%%%%%%%%%%%%%%%%%%%%%%%%%%%%%%%%%
\clearpage
\onecolumn
\appendix       %%% starting appendix
\section*{Appendix A: Python Code}

\lstinputlisting[caption=Code to create solutions, language=Python]{../code/CaseStudy4.py}
\lstinputlisting[caption=Code to generate pretty plots, language=Python]{../code/PrettyPlots.py}

%%%%%%%%%%%%%%%%%%%%%%%%%%%%%%%%%%%%%%%%%%%%%%%%%%%%%%%%%%%%%%%%%%%%%%
\end{document}
