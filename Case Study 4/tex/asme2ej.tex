%%%%%%%%%%%%%%%%%%%%%%%%%%% asme2ej.tex %%%%%%%%%%%%%%%%%%%%%%%%%%%%%%%
% Template for producing ASME-format journal articles using LaTeX    %
% Written by   Harry H. Cheng, Professor and Director                %
%              Integration Engineering Laboratory                    %
%              Department of Mechanical and Aeronautical Engineering %
%              University of California                              %
%              Davis, CA 95616                                       %
%              Tel: (530) 752-5020 (office)                          %
%                   (530) 752-1028 (lab)                             %
%              Fax: (530) 752-4158                                   %
%              Email: hhcheng@ucdavis.edu                            %
%              WWW:   http://iel.ucdavis.edu/people/cheng.html       %
%              May 7, 1994                                           %
% Modified: February 16, 2001 by Harry H. Cheng                      %
% Modified: January  01, 2003 by Geoffrey R. Shiflett                %
% Butchered: October 15, 2014 by John Karasinski                     %
% Use at your own risk, send complaints to /dev/null                 %
%%%%%%%%%%%%%%%%%%%%%%%%%%%%%%%%%%%%%%%%%%%%%%%%%%%%%%%%%%%%%%%%%%%%%%

%%% use twocolumn and 10pt options with the asme2ej format
\documentclass[twocolumn,10pt]{asme2ej}

\usepackage{epsfig} %% for loading postscript figures
\usepackage{amsmath}
\usepackage{graphicx}
\usepackage{grffile}
\usepackage{pdfpages}
\usepackage{algpseudocode}

% Default fixed font does not support bold face
\DeclareFixedFont{\ttb}{T1}{txtt}{bx}{n}{12} % for bold
\DeclareFixedFont{\ttm}{T1}{txtt}{m}{n}{12}  % for normal

% Custom colors
\usepackage{color}
\usepackage{listings}
\usepackage{framed}
\usepackage{caption}
\captionsetup[lstlisting]{font={small,tt}}

\definecolor{mygreen}{rgb}{0,0.6,0}
\definecolor{mygray}{rgb}{0.5,0.5,0.5}
\definecolor{mymauve}{rgb}{0.58,0,0.82}

\lstset{ %
  backgroundcolor=\color{white},   % choose the background color; you must add \usepackage{color} or \usepackage{xcolor}
  basicstyle=\ttfamily\footnotesize, % the size of the fonts that are used for the code
  breakatwhitespace=false,         % sets if automatic breaks should only happen at whitespace
  % breaklines=true,                 % sets automatic line breaking
  captionpos=b,                    % sets the caption-position to bottom
  commentstyle=\color{mygreen},    % comment style
  deletekeywords={...},            % if you want to delete keywords from the given language
  escapeinside={\%*}{*)},          % if you want to add LaTeX within your code
  extendedchars=true,              % lets you use non-ASCII characters; for 8-bits encodings only, does not work with UTF-8
  frame=single,                    % adds a frame around the code
  keepspaces=true,                 % keeps spaces in text, useful for keeping indentation of code (possibly needs columns=flexible)
  columns=flexible,
  keywordstyle=\color{blue},       % keyword style
  language=Python,                 % the language of the code
  morekeywords={*,...},            % if you want to add more keywords to the set
  numbers=left,                    % where to put the line-numbers; possible values are (none, left, right)
  numbersep=5pt,                   % how far the line-numbers are from the code
  numberstyle=\tiny\color{mygray}, % the style that is used for the line-numbers
  rulecolor=\color{black},         % if not set, the frame-color may be changed on line-breaks within not-black text (e.g. comments (green here))
  showspaces=false,                % show spaces everywhere adding particular underscores; it overrides 'showstringspaces'
  showstringspaces=false,          % underline spaces within strings only
  showtabs=false,                  % show tabs within strings adding particular underscores
  stepnumber=1,                    % the step between two line-numbers. If it's 1, each line will be numbered
  stringstyle=\color{mymauve},     % string literal style
  tabsize=4,                       % sets default tabsize to 2 spaces
}

\title{Case Study \# 4: Linear 1D Transport Equation}

\author{John Karasinski
    \affiliation{
  Graduate Student Researcher\\
  Center for Human/Robotics/Vehicle Integration and Performance\\
  Department of Mechanical and Aerospace Engineering\\
  University of California\\
  Davis, California 95616\\
    Email: karasinski@ucdavis.edu
    }
}

\begin{document}
\maketitle

%%%%%%%%%%%%%%%%%%%%%%%%%%%%%%%%%%%%%%%%%%%%%%%%%%%%%%%%%%%%%%%%%%%%%%
\section{Problem Description}
The transport of various scalar quantities (e.g species mass fraction, temperature) in flows can be modeled using a linear convection-diffusion equation (presented here in a 1-D form),
\begin{equation}
\frac{\partial \phi}{\partial t} + u \frac{\partial \phi}{\partial x} = D \frac{\partial \phi^2}{\partial x^2}
\end{equation}
$\phi$ is the transported scalar, $u$ and $D$ are known parameters (flow velocity and diffusion coefficient respectively). The purpose of this case study is to investigate the behavior of various numerical solutions of this equation.

This case study focuses on the solution of the 1-D linear transport equation (above)
for $x$ $\in$ $[0, L]$ and $t$ $\in$ $[0, \tau ]$ (where $\tau = 1/k^2D$) subject to periodic boundary conditions and the following initial condition
\begin{equation}
\phi(x, 0) = sin(kx),
\end{equation}
\noindent with $k = 2\pi/L$ and $L=1$ m. The convection velocity is $u = 0.2$ m/s, and the diffusion coefficient is $D = 0.005$ m$^2$/s.

This problem has an analytical solution \cite{analytic_citation},
\begin{equation}
\Phi(x, t) = exp(-k^2Dt) sin[k(x-ut)].
\end{equation}

Numerical solutions of this problem were created using the following schemes:
\begin{enumerate}
\item FTCS (Explicit) — Forward-Time and central differencing for both the convective flux and the diffusive flux.
\item Upwind — Finite Volume method: Explicit (forward Euler), with the convective flux treated using
the basic upwind method and the diffusive flux treated using central differencing.
\item Trapezoidal — (AKA Crank-Nicholson).
\item QUICK — Finite Volume method: Explicit, with the convective flux treated using the QUICK method and the diffusive flux treated using central differencing.
\end{enumerate}

The following cases are considered:
\begin{equation*}
(C, s) \in \{(0.1, 0.25), (0.5, 0.25), (2, 0.25), (0.5, 0.5), (0.5, 1)\}
\end{equation*}
\noindent where $C = u\Delta t/\Delta x$ and $s = D\Delta t/\Delta x^2$. A uniform mesh for all solvers and cases. The stability and accuracy of these schemes was investigated.


%%%%%%%%%%%%%%%%%%%%%%%%%%%%%%%%%%%%%%%%%%%%%%%%%%%%%%%%%%%%%%%%%%%%%%
\section{Numerical Solution Approach}

%%%%%%%%%%%%%%%%%%%%%%%%%%%%%%%%%%%%%%%%%%%%%%%%%%%%%%%%%%%%%%%%%%%%%%
\section{Results Discussion}

%\begin{table}[htb]
%\begin{center}
%\begin{tabular}{|r r | r r|}
%\hline
%$x (m)$ & $n$ & $NRMS_{xx}$ & $NRMS_{yy}$ \\
%\hline
%3   & 1000 & 14.58\% & 28.29\% \\
%4   & 1000 & 9.01\% & 17.43\% \\
%5   & 1000 & 6.04\% & 8.82\% \\
%100 & 1000 & 4.06\% & 4.19\% \\
%\hline
%\end{tabular}
%\caption{$NRMS$ results from the numerical simulations for the effect of plate length study}
%\label{length_table}
%\end{center}
%\end{table}

%%%%%%%%%%%%%%%%%%%%%%%%%%%%%%%%%%%%%%%%%%%%%%%%%%%%%%%%%%%%%%%%%%%%%%
\section{Conclusion}


%%%%%%%%%%%%%%%%%%%%%%%%%%%%%%%%%%%%%%%%%%%%%%%%%%%%%%%%%%%%%%%%%%%%%%
\nocite{*}
\bibliographystyle{asmems4}
\bibliography{asme2e}

\begin{figure}[thb]
\begin{center}
\includegraphics[width=0.4\textwidth]{figures/CDS.pdf}
\caption{FTCS scheme}
\label{FTCS_scheme}
\end{center}
\end{figure}

\begin{figure}[thb]
\begin{center}
\includegraphics[width=0.4\textwidth]{figures/upwind.pdf}
\caption{Upwind scheme}
\label{Upwind_scheme}
\end{center}
\end{figure}

\begin{figure}[thb]
\begin{center}
\includegraphics[width=0.4\textwidth]{figures/quick.pdf}
\caption{QUICK scheme}
\label{QUICK_scheme}
\end{center}
\end{figure}

\begin{figure}[thb]
\begin{center}
\includegraphics[width=0.5\textwidth]{../code/figures/C=0.1 s=0.25.pdf}
\caption{Results of first case}
\label{base_study}
\end{center}
\end{figure}

\begin{figure}[thb]
\begin{center}
\includegraphics[width=0.5\textwidth]{../code/figures/Error C=0.1 s=0.25.pdf}
\caption{Error for first case}
\label{base_study}
\end{center}
\end{figure}

\begin{figure}[thb]
\begin{center}
\includegraphics[width=0.5\textwidth]{../code/figures/Order FTCS.pdf}
% \includegraphics[width=0.5\textwidth]{../code/figures/Order Upwind.pdf}
% \includegraphics[width=0.5\textwidth]{../code/figures/Order Trapezoidal.pdf}
\includegraphics[width=0.5\textwidth]{../code/figures/Order QUICK.pdf}
\caption{Effective order of accuracy}
\label{base_study}
\end{center}
\end{figure}

\begin{figure}[thb]
\begin{center}
\includegraphics[width=0.5\textwidth]{../code/figures/FTCS 0.1 0.25.pdf}
\includegraphics[width=0.5\textwidth]{../code/figures/FTCS 0.5 0.25.pdf}
\includegraphics[width=0.5\textwidth]{../code/figures/FTCS 0.5 1.pdf}
\caption{FTCS method's transition into instability}
\label{base_study}
\end{center}
\end{figure}

\begin{figure}[thb]
\begin{center}
\includegraphics[width=0.5\textwidth]{../code/figures/QUICK 0.1 0.25.pdf}
\includegraphics[width=0.5\textwidth]{../code/figures/QUICK 0.5 0.25.pdf}
\includegraphics[width=0.5\textwidth]{../code/figures/QUICK 0.5 1.pdf}
\caption{QUICK method's transition into instability}
\label{base_study}
\end{center}
\end{figure}


%%%%%%%%%%%%%%%%%%%%%%%%%%%%%%%%%%%%%%%%%%%%%%%%%%%%%%%%%%%%%%%%%%%%%%
\clearpage
\onecolumn
\appendix       %%% starting appendix
\section*{Appendix A: Python Code}

\lstinputlisting[caption=Code to create plots and solutions, language=Python]{../code/CaseStudy4.py}
\lstinputlisting[caption=Code to generate pretty plots, language=Python]{../code/PrettyPlots.py}

%%%%%%%%%%%%%%%%%%%%%%%%%%%%%%%%%%%%%%%%%%%%%%%%%%%%%%%%%%%%%%%%%%%%%%
\end{document}
