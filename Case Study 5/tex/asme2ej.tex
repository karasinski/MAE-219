%%%%%%%%%%%%%%%%%%%%%%%%%%% asme2ej.tex %%%%%%%%%%%%%%%%%%%%%%%%%%%%%%%
% Template for producing ASME-format journal articles using LaTeX    %
% Written by   Harry H. Cheng, Professor and Director                %
%              Integration Engineering Laboratory                    %
%              Department of Mechanical and Aeronautical Engineering %
%              University of California                              %
%              Davis, CA 95616                                       %
%              Tel: (530) 752-5020 (office)                          %
%                   (530) 752-1028 (lab)                             %
%              Fax: (530) 752-4158                                   %
%              Email: hhcheng@ucdavis.edu                            %
%              WWW:   http://iel.ucdavis.edu/people/cheng.html       %
%              May 7, 1994                                           %
% Modified: February 16, 2001 by Harry H. Cheng                      %
% Modified: January  01, 2003 by Geoffrey R. Shiflett                %
% Butchered: October 15, 2014 by John Karasinski                     %
% Use at your own risk, send complaints to /dev/null                 %
%%%%%%%%%%%%%%%%%%%%%%%%%%%%%%%%%%%%%%%%%%%%%%%%%%%%%%%%%%%%%%%%%%%%%%

%%% use twocolumn and 10pt options with the asme2ej format
\documentclass[twocolumn,10pt]{asme2ej}

\usepackage{epsfig} %% for loading postscript figures
\usepackage{amsmath}
\usepackage{graphicx}
\usepackage{grffile}
\usepackage{pdfpages}
\usepackage{algpseudocode}

% Default fixed font does not support bold face
\DeclareFixedFont{\ttb}{T1}{txtt}{bx}{n}{12} % for bold
\DeclareFixedFont{\ttm}{T1}{txtt}{m}{n}{12}  % for normal

% Custom colors
\usepackage{color}
\usepackage{listings}
\usepackage{framed}
\usepackage{caption}
\captionsetup[lstlisting]{font={small,tt}}

\definecolor{mygreen}{rgb}{0,0.6,0}
\definecolor{mygray}{rgb}{0.5,0.5,0.5}
\definecolor{mymauve}{rgb}{0.58,0,0.82}

\lstset{ %
  backgroundcolor=\color{white},   % choose the background color; you must add \usepackage{color} or \usepackage{xcolor}
  basicstyle=\ttfamily\footnotesize, % the size of the fonts that are used for the code
  breakatwhitespace=false,         % sets if automatic breaks should only happen at whitespace
  % breaklines=true,                 % sets automatic line breaking
  captionpos=b,                    % sets the caption-position to bottom
  commentstyle=\color{mygreen},    % comment style
  deletekeywords={...},            % if you want to delete keywords from the given language
  escapeinside={\%*}{*)},          % if you want to add LaTeX within your code
  extendedchars=true,              % lets you use non-ASCII characters; for 8-bits encodings only, does not work with UTF-8
  frame=single,                    % adds a frame around the code
  keepspaces=true,                 % keeps spaces in text, useful for keeping indentation of code (possibly needs columns=flexible)
  columns=flexible,
  keywordstyle=\color{blue},       % keyword style
  language=Python,                 % the language of the code
  morekeywords={*,...},            % if you want to add more keywords to the set
  numbers=left,                    % where to put the line-numbers; possible values are (none, left, right)
  numbersep=5pt,                   % how far the line-numbers are from the code
  numberstyle=\tiny\color{mygray}, % the style that is used for the line-numbers
  rulecolor=\color{black},         % if not set, the frame-color may be changed on line-breaks within not-black text (e.g. comments (green here))
  showspaces=false,                % show spaces everywhere adding particular underscores; it overrides 'showstringspaces'
  showstringspaces=false,          % underline spaces within strings only
  showtabs=false,                  % show tabs within strings adding particular underscores
  stepnumber=1,                    % the step between two line-numbers. If it's 1, each line will be numbered
  stringstyle=\color{mymauve},     % string literal style
  tabsize=4,                       % sets default tabsize to 2 spaces
}

\title{Case Study \# 5: Two-Species Diffusion-Diurnal Kinetics}

\author{John Karasinski
    \affiliation{
  Graduate Student Researcher\\
  Center for Human/Robotics/Vehicle Integration and Performance\\
  Department of Mechanical and Aerospace Engineering\\
  University of California\\
  Davis, California 95616\\
    Email: karasinski@ucdavis.edu
    }
}

\begin{document}
\maketitle

%%%%%%%%%%%%%%%%%%%%%%%%%%%%%%%%%%%%%%%%%%%%%%%%%%%%%%%%%%%%%%%%%%%%%%
\section{Problem Description}

Chang et al.~\cite{chang1974simulation, byrne1987stiff} have proposed approximate models to describe the chemical kinetics and transport phenomena associated with the dissociation of oxygen (O$_2$) into ozone (O$_3$) and monatomic oxygen (O) in the upper atmosphere. We are considering a one-dimensional version of such a model. The ambient oxygen concentration, $c_3$, is constant, while the concentrations of the two minor species, O and O$_3$, are $c_1(z, t)$ and $c_2(z, t)$, where $z$ is the elevation above the earth's surface in km (here $30 \leq z \leq 50$) and $t$ is time in seconds. Their transport is modeled using a reaction-diffusion equation,
\begin{equation}
\frac{\partial c_i}{\partial t} = \frac{\partial}{\partial z} \left[K(z) \frac{\partial c_i}{dz} \right] + R_i (\vec{c}, t).
\end{equation}
The diffusive term is meant to represent the turbulent vertical transport with
\begin{equation}
K(z) = 10^{-8} \cdot exp(z/5) \quad [\mbox{km}/\mbox{s}],
\end{equation}
 and the chemistry is described using the Chapman mechanism [2].



%%%%%%%%%%%%%%%%%%%%%%%%%%%%%%%%%%%%%%%%%%%%%%%%%%%%%%%%%%%%%%%%%%%%%%
\section{Numerical Solution Approach}

%%%%%%%%%%%%%%%%%%%%%%%%%%%%%%%%%%%%%%%%%%%%%%%%%%%%%%%%%%%%%%%%%%%%%%
\section{Results Discussion}


\begin{table}[tbh]
\begin{center}
\begin{tabular}{| l | r r |}
\hline
Solver & 24 hours & 10 days \\
\hline
dopri5 & 1025     & 10069   \\
bdf    & 1318     & 13237   \\
\hline
\end{tabular}
\caption{Wall clock time, in seconds, to solve to $t=24$ hours and $t=10$ days}
\label{run_time}
\end{center}
\end{table}

\begin{table}[tbh]
\begin{center}
\begin{tabular}{| r | r r |}
\hline
M & $c_{1}$ & $c_{2}$ \\
\hline
  5 & 1.628e-02 & 1.648e-02 \\
 10 & 9.766e-03 & 9.829e-03 \\
 25 & 4.178e-03 & 4.110e-03 \\
 50 & 1.879e-03 & 1.811e-03 \\
 75 & 1.031e-03 & 9.806e-04 \\
100 & 6.923e-04 & 6.098e-04 \\
\hline
\end{tabular}
\caption{NRMS of results at $t=4$ hours compared to results at M = 200 (dopri5 solver)}
\label{dopri5_nrms_table}
\end{center}
\end{table}

\begin{table}[tbh]
\begin{center}
\begin{tabular}{| r | r r |}
\hline
M & $c_{1}$ & $c_{2}$ \\
\hline
  5 & 1.654e-02 & 1.648e-02 \\
 10 & 9.796e-03 & 9.829e-03 \\
 25 & 4.091e-03 & 4.110e-03 \\
 50 & 1.802e-03 & 1.811e-03 \\
 75 & 9.436e-04 & 9.806e-04 \\
100 & 5.982e-04 & 6.098e-04 \\
\hline
\end{tabular}
\caption{NRMS of results at $t=4$ hours compared to results at M = 200 (bdf solver)}
\label{bdf_nrms_table}
\end{center}
\end{table}

\begin{figure}[thb]
\begin{center}
\includegraphics[width=0.5\textwidth]{../code/figures/bdf 86400.0 50 c1.pdf}
\caption{$c_1$ vs. z at $t = $ 0, 2, 4, 6, 7, and 9 hours}
\label{c1_plot}
\end{center}
\end{figure}

\begin{figure}[thb]
\begin{center}
\includegraphics[width=0.5\textwidth]{../code/figures/bdf 86400.0 50 c2.pdf}
\caption{$c_2$ vs. z at $t = $ 0, 2, 4, 6, 7, 9, 12, 18, and 24 hours}
\label{c2_plot}
\end{center}
\end{figure}

\begin{figure}[thb]
\begin{center}
\includegraphics[width=0.5\textwidth]{../code/figures/bdf 864000.0 50 time.pdf}
\caption{$c_1$ and $c_2$ vs. time (from 0 to 10 days) at $z = 40$ km}
\label{40km_plot}
\end{center}
\end{figure}

\begin{figure}[thb]
\begin{center}
\includegraphics[width=0.5\textwidth]{../code/figures/[5, 10, 25, 50, 75, 100, 200].pdf}
\caption{NRMS for M = 5, 10, 25, 50, 75, and 100 compared against M = 200 for both solvers and $c_1$ and $c_2$}
\label{NRMS_plot}
\end{center}
\end{figure}

%%%%%%%%%%%%%%%%%%%%%%%%%%%%%%%%%%%%%%%%%%%%%%%%%%%%%%%%%%%%%%%%%%%%%%
\section{Conclusion}

%%%%%%%%%%%%%%%%%%%%%%%%%%%%%%%%%%%%%%%%%%%%%%%%%%%%%%%%%%%%%%%%%%%%%%
\nocite{*}
\bibliographystyle{asmems4}
\bibliography{asme2e}

%%%%%%%%%%%%%%%%%%%%%%%%%%%%%%%%%%%%%%%%%%%%%%%%%%%%%%%%%%%%%%%%%%%%%%
\clearpage
\onecolumn
\appendix       %%% starting appendix
\section*{Appendix A: Python Code}

\lstinputlisting[caption=Code to create solutions, language=Python]{../code/CaseStudy5.py}
\lstinputlisting[caption=Code to generate pretty plots, language=Python]{../code/PrettyPlots.py}

%%%%%%%%%%%%%%%%%%%%%%%%%%%%%%%%%%%%%%%%%%%%%%%%%%%%%%%%%%%%%%%%%%%%%%
\end{document}
